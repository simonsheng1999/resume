\documentclass{resume}

\newcommand{\en}[1]{#1}
\newcommand{\zh}[1]{}

\zh{\usepackage{xeCJK}}
\zh{\setCJKmainfont{思源宋体}}
\zh{\setCJKsansfont{思源黑体}}
\zh{\setCJKmonofont{思源黑体}}

\begin{document}

\name{\en{Ximing Sheng}\zh{盛熙铭}}
\basicInfo{
      \email{841351034@qq.com} \textperiodcentered\
      \phone{+(86) 15801208963} \textperiodcentered\
      \github[simonsheng1999]{https://github.com/simonsheng1999} \textperiodcentered\
}

\section{\faGraduationCap\ \en{Education}\zh{教育经历}}
\en{\datedsubsection{\textbf{Beijing University of Posts and Telecommunications}, Undergraduate}{08/2017 -- Present}}
\zh{\datedsubsection{\textbf{北京邮电大学}, 在读本科}{2017/08 -- 至今}}
\begin{itemize}
      \item \en{Major: Intelligent Science and Technology, School of Computer. Anticipated graduation date: 06/2021}
            \zh{智能科学与技术,计算机学院,2021年6月毕业}
\end{itemize}

\section{\faUsers\ \en{Work Experience}\zh{工作经历}}
\en{\datedsubsection{\textbf{\href{http://www.4paradigm.com/}{4Paradigm Inc.}}, Beijing, China}{07/2018 -- 08/2018}}
\zh{\datedsubsection{\textbf{\href{http://www.4paradigm.com/}{第四范式(北京)技术有限公司(4Paradigm Inc.)}}}{2018/07 -- 2018/08}}
\en{\role{Data Processing \& Model Build}{Applied Modeling Intern}}
\zh{\role{模型搭建}{应用建模实习生}}
\begin{itemize}
      \item \en{Processing data according to customer requirements, building user models, using tools such as Python/ SQL/Spark.}
            \zh{根据客户需求进行数据处理,建模用户模型,使用Python/Sql/spark等工具
            }
      \item \en{Participate in TOB projects of companies and banks, and build user data model.}
            \zh{参与公司与银行ToB项目,搭建用户数据模型}
\end{itemize}

\en{\datedsubsection{\textbf{\href{https://m.tsingj.com/}{Tsingj Inc.}}, Beijing, China}{10/2020 -- Present}}
\zh{\datedsubsection{\textbf{\href{https://m.tsingj.com/}{华控清交信息科技(北京)有限公司(“华控清交”)}}}{2020/10 -- 至今}}
\en{\role{Cluster Computing}{Cluster R\&D Intern}}
\zh{\role{集群计算}{集群研发实习}}
\begin{itemize}
      \item \en{Participate in migration and optimization of algorithm platform Python to Golang, deployment, testing and optimization of cluster computing.}
            \zh{参与算法平台Python至Golang的迁移与优化,集群计算的部署、测试与优化}
\end{itemize}

\section{\faGithubAlt\ \en{Portfolios}\zh{个人项目}}
\datedsubsection{\textbf{JobShop}}{}
\en{Use genetic algorithm to solve the JobShop problem}
\zh{使用遗传算法解决jobshop问题}
\begin{itemize}
      \item \en{"Introduction to Computing and Programming Course Design" course project and  got a score of 100.}
            \zh{《计算导论与程序设计课程设计》课程项目 成绩100}
      \item \en{Main language and interface used are C + Qt}
            \zh{语言和界面为 C + Qt}
\end{itemize}

\datedsubsection{\textbf{TravelApp}}{\url{https://github.com/simonsheng1999/TravelApp}}
\en{Simulated vehicle inquiry and travel schedule simulation}
\zh{模拟交通工具查询与旅游进度模拟}
\begin{itemize}
      \item \en{Use crawler to get train and air ticket information from 12306 and Ctrip.}
            \zh{使用爬虫从12306、携程爬取车票、机票信息}
      \item \en{Based on HTTP/Threading implementation of remote simulation registration, login. \\
      The system is deployed to the cloud platform to support remote concurrent access.}
            \zh{基于http/threading实现远程模拟注册、登录。系统部署到云平台,支持远程并发访问。}
      \item \en{Main language and interface used are Python + pyQt}
            \zh{语言和界面为 Python + pyQt}
\end{itemize}

\datedsubsection{\textbf{Filestore-Server}}{\url{https://github.com/simonsheng1999/filestore-server}}
\en{Distributed network disk(Graduation Project)}
\zh{\role{分布式网络云盘}{毕业设计}}
\begin{itemize}
      \item \en{Realize network disk fast transmission, block uploading, break point continued transmission and other functions.}
            \zh{实现网盘快传、分块上传、断点续传等功能}
      \item \en{Use Docker\&K8S to deploy the distributed storage cluster based on Ceph.}
            \zh{使用Docker、K8S基于Ceph部署分布式存储集群}

      \item \en{Main language and   frameworks used are Golang + Javascript}
            \zh{语言和界面为 Golang + Javascript}
\end{itemize}

% \section{\faLinkedin\ \en{Personal Experience}\zh{个人经历}}
% \en{\datedsubsection{\textbf{BUPT BYRIO Open Source Community}, Beijing}{2018 -- Present}}
% \zh{\datedsubsection{\textbf{北京邮电大学 BYRIO 开源社区}}{2018 -- 至今}}
% \en{\role{Developer}{Tec Department}}
% \zh{\role{开发者}{技术部}}
% \begin{itemize}
%       \item \en{Participate in the development and maintenance of community products, such as: campus gym appointment system and GPA calculation script.}
%             \zh{参加北邮校园相关产品的开发与维护,如:校园健身房预约系统,北邮教务绩点查询计算脚本}
% \end{itemize}


\section{\faCogs\ \en{Skills}\zh{技能}}
\begin{itemize}[parsep=0.25ex]
      \item \en{\textbf{Programming Language}:
                  \textbf{multilingual} (not limited to any specific language), 
                  experienced in Golang/Python, 
                  comfortable with Java/C/C++}
            \zh{\textbf{编程语言}:
                  \textbf{泛语言}(编程不受特定语言限制),
                  熟悉 Golang/Python,
                  了解 Java/C/C++ 等}

      \item \en{\textbf{Database}:
                  Experienced in frameworks like Scikit-Learn/PyTorch/Keras for machine learning. \\
                  Comfortable with Hadoop, Docker, K8S and other tools in big data and cloud computing.}
            \zh{\textbf{技术框架}:
                  使用过与机器学习相关的 scikit-learn/pytorch/keras等框架\\
                  了解大数据、云计算中 Hadoop、Docker、K8S等工具}

      \item \en{\textbf{Database}:
                  Have used MySQL, TIDB. Comfortable with Redis, etc.}
            \zh{\textbf{数据库}:
                  使用过MySQL、TiDb了解Redis等}

      \item \en{\textbf{Developing Tool}:
                  familiar with Linux-based programming,
                  have experience with team tools like Git, etc.}
            \zh{\textbf{开发工具}:
                  熟悉 Linux,有 Git等团队协作工具的使用经验}

\end{itemize}

\section{\faHeartO\ \en{Awards}\zh{获奖情况}}
\datedline{
      \en{\textit{Mathematical Contest In Modeling and Interdisciplinary Contest In Modeling(MCM/ICM)},{Meritorious Winner(The first prize)}, {2020/04}}
      \zh{\textit{美国大学生数学建模竞赛(MCM/ICM)},{Meritorious Winner(一等奖)} {2020/04}}}

\section{\faInfo\ \en{Miscellaneous}\zh{杂项}}
\begin{itemize}[parsep=0.25ex]
      \item \en{Personal tags: self-driven, quick learner, earnest curiosity, collaborative.}
            \zh{个人标签:自驱动,学习能力强,做事认真,善与协作,保持好奇}
      \item \en{Interests: Distributed System, Big Data, Database and Cloud.}
            \zh{兴趣领域:分布式系统、大数据、数据库、云等}
      \item \en{Selected Courses: OS, Network, Database, Compiler Principle, Machine learning and other AI related basic courses.}
            \zh{主修课程:操作系统、计算机网络、数据库系统原理、编译原理、机器学习等AI相关基础课程}
      % \item \en{Language Level: English CET-6, able to conduct daily conversation and essay reading.}
      %      \zh{语言水平:英语 CET-6,能够进行日常对话和论文阅读}
\end{itemize}

\end{document}
